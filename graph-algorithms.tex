\PassOptionsToPackage{unicode=true}{hyperref} % options for packages loaded elsewhere
\PassOptionsToPackage{hyphens}{url}
%
\documentclass[ignorenonframetext,]{beamer}
\usepackage{pgfpages}
\setbeamertemplate{caption}[numbered]
\setbeamertemplate{caption label separator}{: }
\setbeamercolor{caption name}{fg=normal text.fg}
\beamertemplatenavigationsymbolsempty
% Prevent slide breaks in the middle of a paragraph:
\widowpenalties 1 10000
\raggedbottom
\setbeamertemplate{part page}{
\centering
\begin{beamercolorbox}[sep=16pt,center]{part title}
  \usebeamerfont{part title}\insertpart\par
\end{beamercolorbox}
}
\setbeamertemplate{section page}{
\centering
\begin{beamercolorbox}[sep=12pt,center]{part title}
  \usebeamerfont{section title}\insertsection\par
\end{beamercolorbox}
}
\setbeamertemplate{subsection page}{
\centering
\begin{beamercolorbox}[sep=8pt,center]{part title}
  \usebeamerfont{subsection title}\insertsubsection\par
\end{beamercolorbox}
}
\AtBeginPart{
  \frame{\partpage}
}
\AtBeginSection{
  \ifbibliography
  \else
    \frame{\sectionpage}
  \fi
}
\AtBeginSubsection{
  \frame{\subsectionpage}
}
\usepackage{lmodern}
\usepackage{amssymb,amsmath}
\usepackage{ifxetex,ifluatex}
\usepackage{fixltx2e} % provides \textsubscript
\ifnum 0\ifxetex 1\fi\ifluatex 1\fi=0 % if pdftex
  \usepackage[T1]{fontenc}
  \usepackage[utf8]{inputenc}
  \usepackage{textcomp} % provides euro and other symbols
\else % if luatex or xelatex
  \usepackage{unicode-math}
  \defaultfontfeatures{Ligatures=TeX,Scale=MatchLowercase}
\fi
% use upquote if available, for straight quotes in verbatim environments
\IfFileExists{upquote.sty}{\usepackage{upquote}}{}
% use microtype if available
\IfFileExists{microtype.sty}{%
\usepackage[]{microtype}
\UseMicrotypeSet[protrusion]{basicmath} % disable protrusion for tt fonts
}{}
\IfFileExists{parskip.sty}{%
\usepackage{parskip}
}{% else
\setlength{\parindent}{0pt}
\setlength{\parskip}{6pt plus 2pt minus 1pt}
}
\usepackage{hyperref}
\hypersetup{
            pdfborder={0 0 0},
            breaklinks=true}
\urlstyle{same}  % don't use monospace font for urls
\newif\ifbibliography
\setlength{\emergencystretch}{3em}  % prevent overfull lines
\providecommand{\tightlist}{%
  \setlength{\itemsep}{0pt}\setlength{\parskip}{0pt}}
\setcounter{secnumdepth}{0}

% set default figure placement to htbp
\makeatletter
\def\fps@figure{htbp}
\makeatother


\date{}

\begin{document}

\begin{frame}{What is a graph?}
\protect\hypertarget{what-is-a-graph}{}

A data structure to represent link between objects. A graph is defined
by a set of nodes V and a set of edges E.

We can summarize this definition by the following formula:

\[
G = (E, V)
\]

\begin{block}{What's the difference between a graph and a tree?}

A graph can contain cycles (a node can be visited twice).

\end{block}

\begin{block}{Different type of graphs}

\begin{itemize}
\tightlist
\item
  Acyclic Graph A graph that has no cycle.
\item
  Cyclic Graph A graph that has at least one cycle.
\item
  Directed Graph A graph in which edge has direction. That is the nodes
  are ordered pairs in the definition of every edge.
\item
  Undirected Graph A graph in which edge are not directed. Meaning, the
  edges are defined by an unordered pair of nodes.
\item
  Directed Acyclic Graph A graph that is both directed and acyclic.
\item
  Connected graph Every pair of nodes has a path linking them. Put in
  another way, there are no inaccessible node.
\item
  Disconnected graph A graph in which there is at least one inaccessible
  node.
\end{itemize}

\end{block}

\end{frame}

\begin{frame}{Different way to represent a graph}
\protect\hypertarget{different-way-to-represent-a-graph}{}

There are 2 ways to represent a graph:

\begin{itemize}
\tightlist
\item
  adjacency list For each node, provide a list of other nodes that are
  adjacent to it.
\item
  adjacency matrix A matrix construct by aligning the nodes in the row
  and the columns and putting a value if the nodes are linked by an
  edge.
\end{itemize}

\end{frame}

\begin{frame}{Graph traversal algorithms}
\protect\hypertarget{graph-traversal-algorithms}{}

\begin{block}{DFS (Depth-First Search)}

A graph traversal algorithm in which one start with a root node
(artritrarily choosen) then expore as far as possible along each branch
before backtracking.

\end{block}

\begin{block}{BFS (Breadth-First Search)}

A graph traversal algorithm in which one explore every possible node in
the current depth level before going to the next.

\end{block}

\end{frame}

\begin{frame}{Path finding algorithms}
\protect\hypertarget{path-finding-algorithms}{}

\begin{block}{A* algorithm}

\end{block}

\begin{block}{Dijkstra algorithm}

\end{block}

\end{frame}

\end{document}
